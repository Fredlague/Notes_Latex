\documentclass[letterpaper,12pt,oneside,final]{book}

\input{packages.tex}


\begin{document}
%--------------------------------------------------------------------------------------%

%--------------------------PAGE DE COUVERTURE------------------------------%

% A REMPLIR PAR L'ETUDIANT: 

\newcommand\monPrenom{Frédéric}		%PRENOM
\newcommand\monNom{Laguë}			%NOM
\newcommand\monMatricule{1986131}	%MATRICULE
\newcommand\monGroupe{01}		%GROUPE

%------------------------ Ne pas modifier la ligne suivante --------------%
\input{page_couverture.tex}
%-------------------------------------------------------------------------%


%========================= Début des réponses ============================%


%-----------------------------QUESTION 1 ----------------------------------%
\section*{Question 1}



\begin{enumerate}[a)]

\item % a)
La négation de l'énoncé est:\\
\begin{equation*}
    \text{On dit que} \; g\; \text{n'est pas asymptotique à}\; f \; \text{si} \;
    \; \exists
    \; \varepsilon > 0, \;\forall M \in \mathbb{R}^{+} \; \text{tel que} \; \forall x \geq M, \;
    \text{on a}
\end{equation*}

\begin{equation*}
    \left \vert \frac{f(x)-g(x)}{g(x)} \right \vert > \varepsilon
\end{equation*}
 % A REMPLIR PAR L'ETUDIANT:

\item % b)
On cherche à démonter l'identité:\\
Pour chaque entier positif $n$,

\begin{equation*}
\prod _{i=2}^{n} \left ( 1 - \frac{1}{i^2} \right ) = \frac{n+1}{2n}    
\end{equation*}

D'abord, on démontre que si la l'identité est vrai pour un entier positif $n$, elle est aussi vraie pour $n+1$.

En remplaçant $n$ par $n+1$ dans le membre gauche de l'égalité, on obtient:
\begin{equation*}
    \prod _{i=2}^{n+1} \left ( 1 - \frac{1}{i^2} \right ) = \left (1 - \frac{1}{(n+1)^2} \right )
\cdot \prod _{i=2}^{n} \left ( 1 - \frac{1}{i^2} \right ) 
\end{equation*}

Si l'identité est vraie pour $n$, alors:

\begin{equation*}
    \prod _{i=2}^{n+1} \left ( 1 - \frac{1}{i^2} \right ) = \left (1 - \frac{1}{(n+1)^2} \right ) \left (
    \frac{n+1}{2n} \right )
\end{equation*}

\begin{equation*}
   \Leftrightarrow \; \; \prod _{i=2}^{n+1} \left ( 1 - \frac{1}{i^2} \right ) = \left(1-\frac{1}{n^2+2n+1}\right)\left(\frac{n+1}{2n}\right) = 
    \left(\frac{(n+1)^2-1}{(n+1)^2}\right)\left(\frac{n+1}{2n}\right)
\end{equation*}
 % A REMPLIR PAR L'ETUDIANT:
\begin{equation*}
    \Leftrightarrow \; \; \prod _{i=2}^{n+1} \left ( 1 - \frac{1}{i^2} \right ) = \frac{n^2+2n}{(n+1)(2n)} = \frac{(n+1)+1}{2(n+1)}
\end{equation*}
\end{enumerate}

Ensuite, on démontre que l'identité est vraie pour $n=3$;
\begin{equation*}
    \prod _{i=2}^{3} \left ( 1 - \frac{1}{i^2} \right ) = \left(1-\frac{1}{2^2}\right)
    \left(1-\frac{1}{3^2}\right) = \frac{3}{4} \cdot \frac{8}{9} = \frac{2}{3}
\end{equation*}

\begin{equation*}
    \frac{n+1}{2n} = \frac{3+1}{2\cdot 3} = \frac{2}{3}
\end{equation*}
Similairement, on remarque que l'identité est vraie pour $n=2$\\

L'idendité est donc vrai, pour $n=2$ et $n=3$. De plus, si elle est vraie pour n, elle l'est également pour n+1. Ainsi, elle l'est donc pour tout entier supérieur ou égal à 2.
%-----------------------------QUESTION 2 ----------------------------------%
\newpage
\section*{Question 2}

\subsection*{2.1}

On cherche à démontrer que 
\begin{equation*}
    \left(\bigcup_{i\in \mathcal{I}}A_i\right) \bigcap \left(\bigcup_{i\in \mathcal{J}}B_j\right)
    = \bigcup_{(i,j)\in \mathcal{I}\times\mathcal{J}}A_i \bigcap B_j
\end{equation*}

Pour démontrer que les deux ensembles sont égaux, on doit démontrer que tout élément dans l'ensemble de gauche
est également dans l'ensemble de droite et vice-versa. \\
Supposons
\begin{equation*}
    \exists x \in\left(\bigcup_{i\in \mathcal{I}}A_i\right) \bigcap \left(\bigcup_{i\in \mathcal{J}}B_j\right)
\end{equation*}
Il y a donc deux entiers positifs $k_1\in \mathcal{I}$ et $k_2\in J$ tels que
\begin{equation*}
    x \in A_{k_1}\; \text{et}\; x\in B_{k_2}
\end{equation*}
\begin{equation*}
    \implies x\in A_{k_1} \cap B_{k_2}
\end{equation*}
Or, l'intersection de cette paire d'ensembles est un des ensembles dans l'union du membre de gauche car 
$(k_1, k_2) \in \mathcal{I} \times \mathcal{J}$. Donc, tout élément x du membre de gauche est également dans le membre de droite.

Inversement, supposons
\begin{equation*}
    \exists y \in \bigcup_{(i,j)\in \mathcal{I}\times\mathcal{J}}A_i \bigcap B_j
\end{equation*}
Il y a donc deux cas possibles. Soit (1);
\begin{equation}
    \forall m_1 \in \mathcal{I}\;, \; \exists m_2 \in \mathcal{J}\; \text{tel que} \; x\in A_{m_1} \cap B_{m_2}
\end{equation}
\begin{equation*}
    \implies x \in A_{m_1} \; \text{et} \; x \in B_{m_2}
\end{equation*}
Et donc x est dans les deux ensembles d'unions du membre de gauche, il est nécessairement dans l'intersection.
Ou bien (2);
\begin{equation}
    \forall n_1 \in \mathcal{J}\;, \; \exists n_2 \in \mathcal{I}\; \text{tel que} \; x\in B_{n_1} \cap A_{n_2}
\end{equation}
 
\begin{equation*}
\implies x \in A_{n_2} \; \text{et} \; x \in B_{n_1}
\end{equation*}

Et donc x est dans les deux ensembles d'unions du membre de gauche, il est nécessairement dans l'intersection. \\


\subsection*{2.1}



%-----------------------------QUESTION 3 ----------------------------------%
\newpage
\section*{Question 3}
Soient F, l'ensemble des fonctions $ f: E \rightarrow \{0,1\}$ et A. On cherche à montrer une bijection
$\mathcal{P}(E) \rightarrow F  $.\\

Pour chaque A, un sous-ensemble de E (donc $A\in \mathcal{P}(E)$) on définit la fonction
\begin{equation*}
    g_A: E \rightarrow \{0,1\}
\end{equation*}
telle que 
\begin{equation*}
g_A(x)=
\begin{cases}
    1, &\text{si}\; x \in A\\
    0, & \text{si}\; x\notin A
\end{cases}
\end{equation*}
Ainsi, on peut choisir la fonction $\phi : F \rightarrow A$ entre chaque fonction et ses éléments de $\mathcal{P}(E)$ dont l'image est 1.

\begin{equation*}
    \Phi (f) = \{x \in E \;\vert f(x)=1\}
\end{equation*}

Donc pour chaque élément de P(E), il y a une unique fonction qui associe chaque élément de E dans P(E) à 1 et les autres à 0.
Inversement, pour chaque fonction f vers $\{0,1\}$, il y a un unique sous-ensemble qui contient les éléments associés à 1 par cette fonction.




 % A REMPLIR PAR L'ETUDIANT:


%-----------------------------QUESTION 4 ----------------------------------%
\newpage
\section*{Question 4}



    Soit $X= \{1, 2, 3, \dots\, j\}$, l'ensemble des position de la chambre des j résidents avant l'arrivée de nouveaux invités. 
    Soit $Y = \{1,2,3,\dots, k\}$ la position des résidents (incluants les invités) après l'arrivée des nouveaux invités.
    Pour accommoder tous les résidents et invités, on doit donc choisir une fonction qui associe à chaque résident présent dans une chambre
    une nouvelle chambre où aller.\\
    \begin{enumerate}[a)]
    \item % a)
    
    Soit $n \in \mathbb{N}$ un nombre fini de nouveaux invités. 
    
    On définit donc la fonction: $f_n: X \rightarrow Y\; ;\; x\mapsto n + x$
    où $x\in X$ est un nombre naturel indiquant la position du résident avant l'arrivée. Puisque $n$ est fini
    , les résidents pourront simplement se déplacer de $n$ chambres car f est une bijection, puisque $X$ est dénombrable et $n$ fini. 

    
     % A REMPLIR PAR L'ETUDIANT:
    
    \item % b)
    
    Soit $m \in \mathbb{N}$ un nombre dénombrable de nouveaux invités. 

    On définit donc la fonction $f_m: X\rightarrow Y\;; \;x\mapsto 2x$\\
    où $x\in X$ est un nombre naturel indiquant la position du résident avant l'arrivée. Encore une fois,
    tous les résidents pourront se déplacer car le nombre de nouveaux invités est dénombrable, et donc f est une bijection entre
    les naturels ($X$) et $Y$.


    
     % A REMPLIR PAR L'ETUDIANT:
    
    \item % c)
    Soient $i \in \mathbb{N}$ un nombre dénombrable de trains, ayant chacun un nombre dénombrable $j \in \mathbb{N}$
    de passagers. Puisque le nombre de train et de passager est dénombrable, il est possible d'ordonner les passagers (bijection avec les naturels.)
    Appelons $a_1 \in \mathbb{N}$ le numéro du train et $a_2 \in \mathbb{N}$ le numéro du passager dans le train. Ainsi,
    chaque passager est désigné par un unique couple $(a_1,a_2 \in \mathbb{R}^2)$.
    
    On définit donc la fonction $f: \mathbb{N} \times \mathbb{N} \rightarrow \mathbb{N} \; ;\;
    (a_1,a_2)\mapsto \frac{1}{2}(a_1+a_2)(a_1+a_2+1) + a_2$, qui donnera le numéro de chambre que devra occuper le $a_2$ième nouvel invité du train
    $a_1$. 



    
     % A REMPLIR PAR L'ETUDIANT:
    
    \end{enumerate}




 % A REMPLIR PAR L'ETUDIANT:


%-----------------------------QUESTION 5---------------------------------%
\newpage
\section*{Question 5}

\subsection*{5.1}
Soit $A = \{(x,y) \in \mathbb{R}^2 \;\vert\; 0 < x^2 + y^2 < 1\}$ \\
Toute boule ouverte centrée en un point $(x,y)$ tels que $x^2+y^2=1$, contient des points dans A et à l'extérieur de A.
Donc, $ \partial A = \{(x,y) \in \mathbb{R}^2 \;\vert\; x^2 + y^2 = 1\}$. \\
$\mathring A=A$ car tous les points de l'ensemble ont un voisinage ouvert entièrement dans A.\\
En utilisant le théorème démontré à  l'exercice 1.19; $\overline{A} = \mathring A \cup \partial A$, on a donc que\\
$\overline{A} = \{(x,y) \in \mathbb{R}^2 \; \vert \; x^2 + y^2 \leq 1\}$



\subsection*{5.2}
Tous les points de D sont sur sa frontière car toute boule ouverte sur une des droites de l'union contient des points à l'intérieur
et à l'extérieur de D.
$\mathring D = \emptyset$ car il n'existe aucun point de D qui ait une voisinage qui soit entièrement contenu dans D.
Donc, $\overline{D} = D \cup \{(x,y)\in \mathbb{R}^2 \; \vert \; x=0\}$ car tout point (x,y) de la droite x=0 possède un voisinange ouvert
distinct de x, dans D.


\end{document}
