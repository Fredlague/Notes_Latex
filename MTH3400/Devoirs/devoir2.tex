%--------------------------INITIALISATION DU DOCUMENT---------------------%
%											%														%											%
%                        >>>> Ne pas modifier cette partie <<<<			%
																																					
\documentclass[letterpaper,12pt,oneside,final]{book}
\input{./packages.tex}
\usepackage[]{amsfonts} 


\begin{document}
%--------------------------------------------------------------------------------------%

%--------------------------PAGE DE COUVERTURE------------------------------%

% A REMPLIR PAR L'ETUDIANT: 

\newcommand\monPrenom{Frédéric}		%PRENOM
\newcommand\monNom{Laguë}			%NOM
\newcommand\monMatricule{VotreMatriculeIci}	%MATRICULE
\newcommand\monGroupe{VotreGroupeIci}		%GROUPE

%------------------------ Ne pas modifier la ligne suivante --------------%
\input{page_couverture}
%-------------------------------------------------------------------------%


%========================= Début des réponses ============================%


%-----------------------------QUESTION 1 ----------------------------------%
\section*{Question 1}





\begin{enumerate}[(1)]

\item %(i)
Soit $\varepsilon > 0$. Pour démontrer la convergence de la suite vers $0$, on cherche à démontrer qu'il existe un $n_\varepsilon$ tel que
\[
 n \geq n_\varepsilon \implies d\left( x_n, 0 \right) < \varepsilon
\]
\[
\iff \; \left| x_n - 0 \right| = \left\vert (-1)^{n}\frac{2}{n} \right\vert < \varepsilon
\]
\[
\iff \; \frac{2}{n} < \varepsilon \iff \frac{2}{\varepsilon } < n
\]
Donc, pour $n_\varepsilon > \frac{2}{\varepsilon}$, tous les éléments $x_n$ de la
suite tels que $n\geq n_\varepsilon$ sont contenus dans la boule ouverte, $B_\varepsilon(0)$.
 % A REMPLIR PAR L'ETUDIANT:

\item %(ii)

Soit $\varepsilon > 0$. Pour démontrer la convergence de la suite vers $-\frac{2}{3}$, on cherche à démontrer qu'il existe un $n_\varepsilon$ tel que
\[
n \geq n_\varepsilon \implies d\left( x_n, -\frac{2}{3} \right) < \varepsilon
\] % A REMPLIR PAR L'ETUDIANT:
\[
\iff \; \left| \frac{1-2n}{3n+5} -\left(-\frac{2}{3}\right) \right| = \left|\frac{13}{9n+15} \right|  < \varepsilon    
\]
Donc, pour $n_\varepsilon > \frac{13}{9\varepsilon} $, tous les éléments $x_n$ de la
suite tels que $n\geq n_\varepsilon$ sont contenus dans la boule ouverte, $B_\varepsilon\left(-\frac{2}{3}\right)$.    
\item %(iii)
Soit $\varepsilon > 0$. Pour démontrer la convergence de la suite vers $\frac{1}{\sqrt{3}}$, on cherche à démontrer qu'il existe un $n_\varepsilon$ tel que
\[
n \geq n_{\varepsilon} \implies d\left(x_{n},\frac{1}{\sqrt{3}}\right) < \varepsilon 
\]
\[
\iff\; \left| \frac{\sqrt{n+1}}{\sqrt{3n+1}} - \frac{1}{\sqrt{3}} \right| = \left| \sqrt{\frac{n+1}{3n+1}} - \sqrt{\frac{1}{3}} \right| < \varepsilon  
\]
\[\iff \; \left| \sqrt{\frac{n+1}{3n+1}} \right| + \left| \sqrt{\frac{1}{3}} \right|  \leq \varepsilon 
\]

 % A REMPLIR PAR L'ETUDIANT:


 %0 A REMPLIR PAR L'ETUDIANT:

\end{enumerate}

 % A REMPLIR PAR L'ETUDIANT:




%-----------------------------QUESTION 2 ----------------------------------%
\newpage
\section*{Question 2}


\begin{enumerate}[1)]

\item % a)
Prenons la suite dans $\mathbb{R}^{n}$
 % A REMPLIR PAR L'ETUDIANT:
\[
\left( \vec{x}_k \right)_{k\in \mathbb{N}} = \left(k \sin\left(\frac{k\pi}{2} \right), k \sin\left(\frac{k\pi}{2} \right), \ldots\;,k \sin\left(\frac{k\pi}{2} \right)  \right) 
\]
On peut montrer que cette suite n'est pas bornée. Or, la sous suite
\[ 
    \left( \vec{x}_k \right)_{\{k \in \mathbb{N}\; | \;k\;est\;pair\}}
\] converge effectivement vers $(0, 0, ... , 0)$
\item % b)
Soient les suite;
\[
(\vec{x}_k) = (1,1,1, \ldots\;, 1 - R,\ldots , 1)
\] et \[(\vec{y}_k) = (1, 1, 1, ..., 1 + R, ..., 1)\]
, en alternance pour chaque coordonné. La suite formée de l'union de ces deux suites est alternée, et bornée.
Elle ne converge donc pas, mais est inclue dans la boule ouverte $B_R(e)$
 % A REMPLIR PAR L'ETUDIANT:

\item % c)
Par exemple, la suite 
\[
\left( \vec{x_k} \right)_{k\in \mathbb{N}} = ((-1)^{k}, (-1)^k, \ldots\;, (-1)^{k}) 
\]
est bornée. On y retrouve deux sous suites;
\[
    \left( \vec{a}_k \right) = \left( \vec{x_k} \right)_{\{k\in \mathbb{N} \; k\;est\;pair\}}
\], qui converge vers 1 et 
 % A REMPLIR PAR L'ETUDIANT:
 \[
    \left( \vec{b}_k \right) = \left( \vec{x_k} \right)_{\{k\in \mathbb{N} \; k\;est\;impair\}}
\], qui converge vers -1.
\item Une suite convergente vers $x$ a toujours un nombre fini d'éléments dehors de $B_{\varepsilon}(x)$ 
Ainsi, les éléments de la suite qui ne sont pas bornées dans la boule ouverte sont dénombrables. On peut majorer et minorer chacun de ces éléments,
et la suite doit donc être bornée.
\end{enumerate}

%-----------------------------QUESTION 3 ----------------------------------%
\newpage
\section*{Question 3}


\begin{enumerate}[a)]

\item % a)
Soit $U$ un ouvert dans $\mathbb{R}^n$ et soit $u \in U$. Par définition d'un ensemble ouvert;
\[
\exists \varepsilon_u > 0 \text{tel que } B_{\varepsilon_u}(u)
\]
pour chaque $u \in U$. Alors;
\[
U = \bigcup_{u\in U} B_{\varepsilon_u}(u)
\]
Selon le Théorème 1.6, un union d'ensembles ouvert est ouvert.\\
Supposons que $U$ soit un union de boules ouvertes, mais pas un ouvert.\\
Donc, $\exists a \in U$ tel que $\forall \varepsilon > 0$, $B_{\varepsilon}(a)\; \backslash\; U $ n'est pas vide.\\
Donc, $\exists u \in U$ et $\varepsilon_u > 0$ tels que $a \in B_{\varepsilon}(u)$\\
On peut donc trouver une boule ouverte plus petite autour de a;\\
$\exists \varepsilon_a > 0$ tel que $B_{\varepsilon_a}(a) \subset B_{\varepsilon_u}(u)$.
Donc $U$ est ouvert.

Ce qui signifie que $U$ est un ouvert dans $\mathbb{R}^n$ seulement si il est un union de boules ouvertes.

 % A REMPLIR PAR L'ETUDIANT:

\item % b)

L'ensemble des rationnels est un sous ensemble dénombrable, car il existe une bijection avec les naturels
De plus, il est dense dans $\mathbb{R}$, comme indiqué au Chapitre 1.

% A REMPLIR PAR L'ETUDIANT:
\item 

\end{enumerate}


%-----------------------------QUESTION 4 ----------------------------------%
\newpage
\section*{Question 4}


\begin{enumerate}[a)]

\item % 
   Soit $ \varepsilon > 0 $
 % A REMPLIR PAR L'ETUDIANT:
. Alors $f_n$ converge vers $f$ si, 

\[
n \geq n_\varepsilon\; \implies\; d \left( f_n(x), f(x) \right) < \varepsilon,\;  \forall x
\in [0,1]
\]
En prenant la fonction $f: [0,1] \mapsto \mathbb{R} $  , définie comme; 
\[
f(x) = \begin{cases}
    0 & , \text{ si } x\in [0,1[\\
    1 & , \text{ si } x=1\\
\end{cases}
\]
On a donc
\[
d\left( f_n(x), f(x) \right) = \begin{cases}
    \left| \frac{x^{n}}{(x^{n}+1)} \right| &  \text{, si } x \in [0,1] \\
    \big\vert \frac{x^{n}}{x^{n}+1}  \big\vert & \text{, si } x = 1
\end{cases}    
\]
$Pour\;le\;cas\;x= 1,\;$ on\; a
\[
d\left( f_n(1), f(1) \right) = \left \vert \frac{1^{n}}{1^{n}+1}-1 \right \vert = 0 
\]
On a donc que $\forall n_{\varepsilon} \in \mathbb{N}$, $\forall \varepsilon > 0$, 
\[
n \geq n_\varepsilon\; \implies\; d \left( f_n(x), f(x) \right) < \varepsilon 
\]
$Pour\;le\;cas\;x \neq 1$ , on a
\[
    d\left( f_n(x), f(x) \right) = \left \vert \frac{x^{n}}{x^{n}+1}-0 \right \vert =  \frac{x^{n}}{x^{n}+1} < \varepsilon  
\]
\[
\iff \; x^{n} < \frac{\varepsilon}{\varepsilon -1}  
\]
\[
\iff \; n > \frac{\ln\left( \frac{\varepsilon}{1- \varepsilon } \right)}{\ln x}
\]

On doit donc distinguer le cas où $x=0$.\\ On trouve également que $d(f_n(0),f(0)) = 0. $\\ 
 Ainsi, on a donc que pour $n_\varepsilon > \frac{\ln\left( \frac{\varepsilon}{1- \varepsilon } \right)}{\ln x}$;
 \[
    n \geq n_\varepsilon\; \implies\; d \left( f_n(x), f(x) \right) < \varepsilon,\;  \forall x \in [0,1[
 \]
On a donc une fonction $f(x)$ vers laquelle $f_n(x)$ converge $\forall x \in [0,1]$ car
\[
\forall \varepsilon > 0,\; \exists n_\varepsilon \in \mathbb{N} \text{ tel que }, 
n\geq n_{\varepsilon} \text{ implique } d \left( f_n(x), f(x) \right) < \varepsilon,\;  \forall x
\in [0,1] 
\]
\item % b) 

 % A REMPLIR PAR L'ETUDIANT:



\end{enumerate}


\end{document}
